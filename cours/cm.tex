\documentclass[12pt, a4paper]{article}
\usepackage[francais]{babel}
\usepackage{caption}
\usepackage{graphicx}
\usepackage[T1]{fontenc}
\usepackage{listings}
\usepackage{geometry}
\usepackage{minted}
\usepackage{array,multirow,makecell}
\usepackage[colorlinks=true,linkcolor=black,anchorcolor=black,citecolor=black,filecolor=black,menucolor=black,runcolor=black,urlcolor=black]{hyperref}
\setcellgapes{1pt}
\makegapedcells
\usepackage{fancyhdr}
\pagestyle{fancy}
\lhead{}
\rhead{}
\chead{}
\rfoot{\thepage}
\lfoot{Martin Baumgaertner}
\cfoot{}
\renewcommand{\footrulewidth}{0.4pt}
\renewcommand{\headrulewidth}{0.4pt}
\renewcommand{\listingscaption}{Code}
\renewcommand{\listoflistingscaption}{Table des codes}
% \usepackage{mathpazo} --> Police à utiliser lors de rapports plus sérieux

\begin{document}
\begin{titlepage}
	\newcommand{\HRule}{\rule{\linewidth}{0.5mm}} 
	\center 
	\textsc{\LARGE iut de colmar}\\[6.5cm] 
	\textsc{\Large R312}\\[0.5cm] 
	\textsc{\large Année 2022-23}\\[0.5cm]
	\HRule\\[0.75cm]
	{\huge\bfseries Communication professionnelle}\\[0.4cm]
	\HRule\\[1.5cm]
	\textsc{\large martin baumgaertner}\\[6.5cm] 

	\vfill\vfill\vfill
	{\large\today} 
	\vfill
\end{titlepage}
\newpage
\tableofcontents
\newpage
\section{CM 1 - 12 septembre 2022}

Les différents sujets pour l'évaluation de l'exposé :\\
\begin{itemize}
    \item Évolution de la sécurité des systèmes d'information
    \item Histoire des télécoms
    \item Histoire de l'informatique 
    \item Le bon usage des données informatiques, des droits d'auteurs et du 
    droit à l'image 
\end{itemize}

\subsection{Introduction}
\subsubsection{Définition}
Action de communiquer, de transmettre des informations ou des connaissances
à quelqu'un ou, s'il y a échange, de les mettre en commun.\\

\subsubsection{La communication}
Il y plusieurs styles de communication : interpersonnelles et de masse. 


\end{document}