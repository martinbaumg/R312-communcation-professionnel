\documentclass[12pt, a4paper]{article}
\usepackage[francais]{babel}
\usepackage{caption}
\usepackage{graphicx}
\usepackage[T1]{fontenc}
\usepackage{listings}
\usepackage{geometry}
\usepackage{minted}
\usepackage{array,multirow,makecell}
\usepackage[colorlinks=true,linkcolor=black,anchorcolor=black,citecolor=black,filecolor=black,menucolor=black,runcolor=black,urlcolor=black]{hyperref}
\setcellgapes{1pt}
\makegapedcells
\usepackage{fancyhdr}
\pagestyle{fancy}
\lhead{}
\rhead{}
\chead{}
\rfoot{\thepage}
\lfoot{Martin Baumgaertner}
\cfoot{}
\renewcommand{\footrulewidth}{0.4pt}
\renewcommand{\headrulewidth}{0.4pt}
\renewcommand{\listingscaption}{Code}
\renewcommand{\listoflistingscaption}{Table des codes}
% \usepackage{mathpazo} --> Police à utiliser lors de rapports plus sérieux

\begin{document}
\begin{titlepage}
	\newcommand{\HRule}{\rule{\linewidth}{0.5mm}} 
	\center 
	\textsc{\LARGE iut de colmar}\\[6.5cm] 
	\textsc{\Large R312}\\[0.5cm] 
	\textsc{\large Année 2022-23}\\[0.5cm]
	\HRule\\[0.75cm]
	{\huge\bfseries Communication professionnelle}\\[0.4cm]
	\HRule\\[1.5cm]
	\textsc{\large martin baumgaertner}\\[6.5cm] 

	\vfill\vfill\vfill
	{\large\today} 
	\vfill
\end{titlepage}
\newpage
\tableofcontents
\newpage
\section{CM 1 - 12 septembre 2022}

Les différents sujets pour l'évaluation de l'exposé :\\
\begin{itemize}
    \item Évolution de la sécurité des systèmes d'information
    \item Histoire des télécoms
    \item Histoire de l'informatique 
    \item Le bon usage des données informatiques, des droits d'auteurs et du 
    droit à l'image 
\end{itemize}

\subsection{Introduction}
\subsubsection{Définition}
Action de communiquer, de transmettre des informations ou des connaissances
à quelqu'un ou, s'il y a échange, de les mettre en commun.\\

\subsubsection{La communication}
Il y plusieurs styles de communication : interpersonnelles et de masse. 

\newpage
\section{CM 2 - 19 septembre 2022}
Comment réduire l'effet des filtres ? Adapter son discous, son langage, la puissance
de la voix, l'articulation, la gestuelle, le rythme, la durée, la distance. \\

Il y a plusieurs types de communication :
\begin{itemize}
	\item La communication verbale
	\item La communication non verbale
	\item La communication écrite
	\item La communication visuelle\\
\end{itemize}

et il y a 3 types de communication organisationnelle :
\begin{itemize}
	\item La communication ascendante
	\item La communication descendante
	\item La communication latérale
\end{itemize}

\newpage

\section{CM 3 - 7 novembre 2022}
\subsection{La communication écrite}
\subsubsection{La bonne tenue en présentation}
\begin{itemize}
	\item La tenue vestimentaire
	\item La tenue de la salle
	\item La tenue du support
\end{itemize}

\subsubsection{La bonne tenue en rédaction}
\begin{itemize}
	\item La mise en page
	\item La typographie
	\item La mise en forme
	\item La mise en couleur
\end{itemize}

\newpage

\section{CM 4 - 14 novembre 2022}
\subsection{Bien faire son CV}

Le CV est un document qui permet de présenter ses compétences et ses expériences
professionnelles. Il est souvent accompagné d'une lettre de motivation.\\

\subsubsection{La mise en page}

Il faut faire attention à la mise en page du CV. Il faut que le CV soit lisible
et qu'il soit bien structuré. Il faut aussi que le CV soit bien présenté. Car
il ne faut pas oublier que le recruteur regardera en moyenne 3 secondes le CV
et suite à cela, il décidera si il va le lire ou non.\\





\end{document}